In order to adapt HopsFS to run in different 

\subsection{System behavior}
Detail the desired behavior of the system both in normal operating conditions and during data-center failure/network partition.

\subsection{Split brain detection}
Failure detector: 
\textit{http://dl.acm.org/citation.cfm?id=226647}

\subsection{Database replication}
Explain how replication was configured on the MySQL Cluster clusters.
\subsubsection{Access patterns}
Detail access patterns to different tables on the database and the required consistency characteristics.
\begin{itemize}
    \item access on master: leases and leader election require consistency
    \item conflict avoidance: if the primary key is a numeric ID we can avoid conflicts by assigning even/odd ids to different clusters
    \item conflict handling: for inodes we can handle conflict by renaming files
\end{itemize}

Pending problems: handling of leases during DC failure/network partition is still being discussed.

\subsubsection{MySQL HA Setup}
Explain how to make async replication hihghly available (in current model, replication channel failure is interpreted as network partition).

\subsection{Leader election}
Modifications to the leader election process to detect network partitions.
\subsection{Data Access Layer}
Changes to the data access layer to accomodate for the new data access patterns described above.
\subsection{Block placement policy}
Changes to the block placement policies and block recovery both during normal operations and during network partition/datacenter failure.