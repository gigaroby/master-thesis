\subsection{Block management}
Aside from managing metadata in a replicated environment, a geographically replicated storage system also needs to manage file content in such a way that, during a split brain, the separated clusters are capable of serving all client requests.
As previously mentioned, HDFS and HopsFS, store file content in \emph{blocks} which are managed by data nodes.
A file can span arbitrarily many blocks which have a configurable maximum size, by default 150 megabytes.
Blocks are immutable once they are marked as finished and only the last block in a file can be modified.
Adding content at the end of a block is the only modification allowed.

In order to maintain availability of blocks in the face of data node failure, HopsFS supports two different replication schemes: block replication and erasure coding.

\paragraph{In block replication,} the system maintains multiple copies of the same block on different data nodes.
The copies, called replicas, are distributed among data nodes according to a configurable placement policy, which aims to minimize the number of blocks which are unavailable as a result of component failure, be it machine or switch.
In case one of the replicas is permanently lost, the leader namenode instructs data nodes to re-replicate the block, returning the amount of replicas to the specified number, three by default.

\paragraph{Erasure coding} is a radically different concept than whole block replication.
Instead of creating entire copies of the blocks, erasure coding computes new parity blocks from the original blocks.
Both the number of source and output blocks are configurable, and the output blocks are called parity blocks.
Assuming $N$ source blocks, 10 for example, and $M$ parity blocks, 5 for example, the 10 original blocks can be reconstructed using any combination of the $N + M$ blocks now available.
The parity blocks form a new file, which is stored in a different directory than the original file.
Block placement for erasure coding blocks is handled by the \emph{erasure coding manager} which is described in \cite{grohsschmiedt2014making}.
In case one of the blocks for a erasure coded file fails, the system needs to regenerate either the original block or the parity block, which requires a full read of $N$ of the blocks and it is accomplished through a mapreduce job.

In this work we only consider whole block replication but we plan to implement erasure coding schemes in later iterations of the project.

\subsubsection{Placement policy}
Given that the goal for the project is to allow both geographical regions to operate independently in case of split brain, each region requires a complete set of blocks from all locations.
In order to obtain this, the block placement policy needs to be aware of the existence of multiple zones, which are considered separate failure domains.
The existing hierarchy for failure domains only considers machine and rack but the modified version will also include a third level: \emph{geographical zone}.

While this solves the issue of placing blocks in the correct datanodes, there remains the issue of the number of replicas to create.
The default value of three creates imbalance, by assigning two replicas in the zone where the block was created and only one in the other zone.
By using a replica value of four two replicas are assigned to every zone, ensuring that both zones have the same amount of blocks.

The final problem regarding the placement policy is the handling of split brain scenarios.
Without further adaptations, a split brain scenario would lead the cluster to believe that half of all the replicas in the cluster are missing, forcing the leader namenode to re-replicate all the blocks an additional two times.
Aside from creating a very large amount of network load between datanodes the result of such re-replication would be discarded as soon as the temporary split brain scenario is resolved.
In order to avoid spurious re-replication, we modify the amounts of replicas to two during split brain scenarios.
By setting the value to two, we avoid any re-replication of existing blocks and we only create two replicas for new blocks.
When the two clusters are merged, the replica value is once again increased to four, and the normal background re-replication tasks will create the necessary replicas in the other zone for blocks created during the split brain.

\subsubsection{}